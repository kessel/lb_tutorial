\chapter{The LBM in brief}

\subsection*{Linearized Boltzmann equation}

Here we want to repeat a few very basic facts about the LBM. 
You will find much better introductions in various books and
articles, e.g. \cite{succi, duenweg}. It will however help clarifying 
our choice of words and we will eventually say something about the 
implementation. It is very loosely written, with the goal that
the reader understands how the LBM works and what \ES{} does without
being precise with all details. 

The LBM essentially consists in solving a fully discretized
version of the linearized Boltzmann equation. The Boltzmann equation
describes the time evolution of the one particle distribution
function, which is the probability to find a molecule in a phase
space volume $\left(x,p\right)\d x \d p$. In the context of
LB, it is useful to apply a slightly different normalization of
the one particle distribution function: The function $f$ is normalized
so that the integral over the whole phase space is the total 
number of particles:
\begin{equation*}
  \int f\left(x,p\right)\d x \d p = N,
\end{equation*}
so that the quantity f$\left(x,p\right)\d x \d p$ corresponds
to the number of particles in this particular cell of the phase
space, the population. \\

\subsection*{Discretization}

The LBM discretizes the Boltzmann equation not only in real
space (the lattice!) and time, but also the velocity space is discretized 
in a surprisingly small number of velocities, in 3D typically
19, sometimes more, rarely less. 
Mostly we will refer to the three dimensional model with a discrete
set of 19 velocities, which is conventionally called D3Q19.
These velocities
are chosen so that they correspond the movement from one lattice
node to another in one time step. A two step scheme is used to transport
information through the system: In the streaming step
the particles ( in terms of populations ) are transported
to the cell where they corresponding velocity points to. 
In the collision step, the distribution functions
in each cell are relaxed towards the local thermodynamic
equilibrium. This will be described in more detail below.

The important hydrodynamic properties, the density, the fluid momentum density,
the pressure tensor can be calculated quite straightforward from
the populations: They just correspond the to the moments
of the distribution function: 
\begin{align}
  \rho &= \sum f_i \\
  \vec{j} = \rho \vec{u} &= \sum f_i \vec{c_i} \\
  \Pi^{\alpha \beta} &= \sum f_i \vec{c_i}^{\alpha}\vec{c_i}^{\beta} 
  \label{eq:fields}
\end{align}
Here the Greek indices denotes the cartesian axis and the
Latin indices indicate the number in the disrete velocity set.
We will occasionally refer to these quantities as the hydrodynamic fields.
Note that the pressure tensor is symmetric.
It is easy to see that these equations are linear transformations
of the $f_i$ and that they carry the most important information. They
are 10 independent variables, but this is not enough to store the
full information of 19 populations. Therefore 9 additional quantities
are introduced. Together they form a different basis set of the
19-dimensional population space. This new basis set is called
the modes $m_i$. Indeed the most practical basis is a little bit different:
It contains the trace of the pressure tensor and two orthogonal 
vectors instead of the three diagonal elements of the pressure 
tensor. The 9 extra modes are referred to as kinetic modes or
ghost modes. It is possible to explicitly write down the 
base transformation matrix, and its inverse and in the \ES{}
LBM implementation this basis transformation is made for every
cell in every LBM step. It is possible to write a code that does not
need this basis transformation, but it has been shown, that this
only costs 20\% of the computational time and allows for 
larger flexibility.

\subsection*{The second step: collision}

The part that is now still missing is the collision part, where
the actual physics happens. For the LBM it is assumed that
the collision process linearly relaxes the populations to the local
equilibrium, thus that it is a linear (=matrix) operator 
acting on the populations in each LB cell. It should conserve 
the particle number and the momentum. At this point it is clear
why the mode space is helpful. A 19 dimensional matrix that
conserves the first 4 modes (with the eigenvalue 1) is diagonal in the
first four rows and columns.
Some struggling with lattice symmetries shows that four independent
variables can be chosen to characterize the linear relaxation
process which are still compatible with the prerequisite that
the fluid should be isotropic: Two of them are closely related to 
the shear and bulk viscosity of the fluid, and two of them
do not have a direct physical equivalent. They are just called
relaxation rates of the kinetic model.


The equilibrium distribution to which the populations relax 
is obtained from maximizing the information entropy 
$\sum f_i \log f_i$ under the constraint that the density
and velocity are assigned their particular instantaneous 
values. This leads to the following expression for the
equilibrium populations:

In mode space the equilibrium distribution is again very 
easily formulated: The modes 5-19 have the value 0 in equilibrium.

Then the collision operator is diagonal:
\begin{align*}
  m^\star_i &= \gamma_i m_i  \\
  \gamma_1=\dots=\gamma_4&=1 \\
  \gamma_5&=\gamma_\text{b} \\
  \gamma_6=\dots=\gamma_{10}&=\gamma_\text{s} \\
  \gamma_{11}&=\gamma_\text{even} \\
  \gamma_{12}=\dots = \gamma_{19}&=\gamma_\text{odd} \\
\end{align*}

To include hydrodynamic fluctuations of the fluid, 
random fluctuations are added to the modes $4\dots 19$ on every LB node so that
the LB fluid temperature is well defined and the corresponding
fluctuation formula, according to the fluctuation dissipation theorem holds.
An extensive discussion of this topic is found in \cite{schiller08a}

\subsection*{Particle coupling}

Particles are coupled to the LB fluid with the force coupling:
The fluid velocity at the point where a particle is, is calculated 
by a multilinear interpolation and a force is applied on the particle
that is proportional to the velocity difference between particle 
and fluid:
\begin{equation}
  \vec{F} = - \gamma \left(v-u\right) 
\end{equation}
The opposite force is distributed on the surrounding LB nodes. Additionally
a random force is added to maintain a constant temperature, again according
to the fluctuation dissipation theorem.


