\chapter{Electro-osmotic flow in a slit pore}
Electro-osmotic flow (EOF) is the motion of water (or another liquid)
induced by an electric field. It can occur e.g. in porous media,
in synthetic capillaries and in vicinity of charged surfaces.
Charged objects in an electrolyte solution attract ions of one
species and repel ions of the other species, which gives rise
to a net charge density in its neighbourhood. If an external
electric field is applied, these ions are accelerated in the direction
of the electric field (or oppositely if negatively charged) which
causes also an acceleration of the surrounding water. In regions
with zero net charge, the force on the fluid exerted by both ion
species cancels, thus charged interfaces are necessary.

Conceptually electro-osmotic flow is closely related to electrophoresis
where a charged object, e.g. a polyelectrolyte, is moved by an
electric field and the surrounding counterions create a flow field
in the opposite direction. 

In this exercise the electrokinetic equations, that allow for a classical
description of the phenomenon, are introduced and you will learn
how to simulate this effect with \ES{} with the LBM. The special case of
planar charged walls in the regime of low salt concentration can
be solved analytically and you will see that we can reproduce the
classical results quite well, but you will also learn about the
deficiencies of both approaches. We will concentrate on the case
where only one species of ions (counterions) is present. The generalization
to multiple species however is straightforward.

\section{The electrokinetic equations}
We want to describe a system in which ions can diffuse under an
applied field embedded in a fluid. We therefore assume that a 
linear convection diffusion is valid:
\begin{equation}
  \vec{j}=-D \nabla c + \mu ze c \vec{E} + c \vec{u}
\end{equation}
Here $j$ corresponds to the ion flux density, $D$ corresponds 
to the diffusion coefficient of the ions,
$c$ to their concentration, $\mu$ to their (electrophoretic)
mobility, $\vec{E}$ to the local electric field and $\vec{u}$
to the fluid velocity.

The fluid fulfills the incompressible Navier-Stokes equation:
Here the term $c vec{E}$ appears as source term due
to the acceleration of the fluid caused by the ions. 

For the electrostatic potential we make the following
mean field approximation: The electric potential is 
caused not by single ions, but their density. This means
every ion is not exposed to the instantaneous potential
but the smeared out potential of all other ions. Then the Poisson
equation reads as:
\begin{equation}
  \Delta \Phi = -c/\varepsilon
  \label{asdf}
\end{equation}
We will later
see that this approximation can be overcome if we move
from a continuum description to a molecular dynamics simulation
of explicit ions.

This set of coupled partial differential equations is called
the electrokinetic equations. 

\section{The slit pore geometry}
We want to investigate the simplest case where EOF occurs:
The flow water through the volume between two parallel
charged planes in the $xy$-plane. We assume that the planes are infinitely
extended in the directions parallel to the plane and that
the number of ions exactly cancels the charge of both planes
and that the external electric field is exerted in $x$ direction
and that the position of the plates is at $z=\pm l/2$.
The translational invariance allows to greatly simplify the
electrokinetic equations:
\begin{eqnarray}
  u_z = u_y = 0 \\
  j_z = 0 \doublerightarrow +D \nabla c + \mu ze c \partial_z \Phi = 0
  \partial_z \Phi = -c/\varepsilon \\
  \partial_z^2 u_x = zeEc
  \label{EOF}
\end{eqnarray}
This set of ordinary differential equations is almost decoupled, 
and the ion concentration profile $c\left(z\right)$ is independent
of the fluid velocity. With the Einstein relation $D=\mu kT$ it the
Poisson equation can be written in the following well known form:
\begin{equation}
  \partial_z^2 Phi = c_0 \exp{\frac{-ze\Phi}{kT}}
\end{equation}
which is the one-dimensional Poisson-Boltzmann equation. One can say:
The density of the ions is proportional to their associated
Boltzmann factor. The factor $c_0$ appears as integration constant
and can be chosen to assure charge neutrality.

For planes with charge density $\sigma$ this can be solved by:
Here the parameter $asdf$ has to fulfill the following transcendental
equation:
Integrating the charge density twice yield the fluid flow field
in the direction parallel to the applied field:
Here the integration constants were chosen so that no-slip boundary
conditions are fulfilled a $\pm l/2$.
\section{Simulating the system with \ES{}}
To simulate the system with on the LB+MD we have almost all
ingredients onboard: In previous tutorials we have learned how
to deal with charged particles, in the last section
we have learnt how to implement planar walls.



Classical description:
Poisson-Boltzmann equation
Source term in Stokes equation

In the geometry we are 

