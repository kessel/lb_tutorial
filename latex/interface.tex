

\chapter{The LB interface in \ES{}}
In \ES{} the LB scheme and the MD scheme are not synchronized: In one
LB time step typically several MD steps are performed. This allows to speed
up the simulations and is adjusted with the parameter \lstinline|tau|.
\ES{} has three main commands for the LB module: 
 \lstinline|lbfluid|,  \lstinline|lbnode|, and  \lstinline|lb_boundary|.
 \lstinline|lbfluid| is mainly used to set up parameters and does everything that
concerns the whole fluid.  \lstinline|lbnode| involves readout and manipulation of
single LB cells.  \lstinline|lb_boundary| allows to set boundaries, currently only
the bounce back boundary method is implemented to model
no-slip walls. Additionally the command  \lstinline|thermostat| \lstinline|lb| is used to set
the temperature. 

Important Notice: All commands of the LB interface use
MD units. This is convenient, as e.g. a particular 
viscosity can be set and the LB time step can be changed without
altering the viscosity. On the other hand this is a source
of a plethora of mistakes: The LBM is only reliable in a certain 
range of parameters (in LB units) and the unit conversion
may take some of them far out of this range. So note that you always
have to assure that you are not messing with that!

One brief example: a certain velocity may be 10 in MD units.
If the LB time step is 0.1 in MD units, and the lattice constant
is 1, then it corresponds to a velocity of 1 in LB units. 
This is the maximum velocity of the discrete velocity set and therefore
causes numerical instabilities like negative populations.

Now the commands in more detail.

\subsection*{The \lstinline|lbfluid| command}
The \lstinline{lbfluid} command sets global parameters of the LBM. Every
parameter is given in the form \lstinline|lbfluid name value|. 
All parameters except for \lstinline|gamma_odd| and  \lstinline|gamma_even|
are given in MD units. All parameters except for \lstinline|ext_force| accept
one scalar floating point argument. \\

\begin{tabular}{p{0.2\columnwidth}p{0.5\columnwidth}}
\lstinline|dens| & The density of the fluid.\\
\lstinline|grid| & The lattice constant of the fluid. It is used to determine the number of LB nodes 
per direction from \lstinline|box_l|. {\em They have to be compatible.} \\
\lstinline|visc| & The kinematic viscosity \\
\lstinline|tau| & The time step of LB. It has to be larger than the MD time step. \\
\lstinline|friction| & The friction coefficient $\gamma$ for the coupling scheme. \\
\lstinline|ext_force| & An external force applied to every node with three components. \\
\lstinline|gamma_odd| & Relaxation parameter for the odd kinetic modes. \\
\lstinline|gamma_even| & Relaxation parameter for the even kinetic modes.
\end{tabular} \\

A good starting point for an MD time step of 0.01 is the command line
\vspace{0,2cm}
\begin{lstlisting}[ numbers=none]
lbfluid grid 1.0 dens 10. visc 1. tau 0.1 friction 20. gamma_odd 0. gamma_even 0.
\end{lstlisting}
\vspace{0,2cm}

In combination with walls, it could also be useful to set \lstinline|gamma_odd| and
\lstinline|gamma_even| to -0.7. E.g. 
\vspace{0,2cm}
\begin{lstlisting}[ numbers=none]
lbfluid grid 1.0 dens 10. visc 1. tau 0.1 friction 20. gamma_odd -0.7 gamma_even -0.7 ext_force 0.01 0. 0.
\end{lstlisting}
\vspace{0,2cm}

\subsection*{The \lstinline|lbnode| command}
The \lstinline|lbnode| command allows to inspect and modify single LB nodes The
general syntax is:
\vspace{0,2cm}
\begin{lstlisting}[ numbers=none]
lbnode X Y Z command arguments
\end{lstlisting}
\vspace{0,2cm}
Note that the indexing in every direction starts with 0. The possible commands are:\\
\begin{tabular}{p{0.2\columnwidth}p{0.5\columnwidth}}
  print & Print one or several quantities to the TCL interface.\\
  set & Set one quantity to a particular value (can be a vector)\\
\end{tabular}\\
For both commands you have to specify what quantity should be printed
or modified. Print allows for the following arguments: \\
\begin{tabular}{p{0.2\columnwidth}p{0.5\columnwidth}}
  \lstinline|rho|\ & the density (scalar). \\
  \lstinline|u| & the fluid velocity (three floats: $u_x$, $u_y$, $u_z$) \\
  \lstinline|pi| & the fluid velocity (six floats: $\Pi_{xx}$, $\Pi_{xy}$, $\Pi_{yy}$, $\Pi_{xz}$,  $\Pi_{yz}$,  $\Pi_{zz}$) \\
  \lstinline|pi_neq| & the nonequilbrium part of the pressure tensor, components as above. \\
  \lstinline|pop| & the 19 population (check the order from the source code please).
\end{tabular} \\
Example:
The line
\begin{lstlisting}[ numbers=none]
puts [ lbnode 0 0 0 print u ]
\end{lstlisting}
prints the fluid velocity in node 0 0 0 to the screen.
The command \lstinline|set| allows to change the density or fluid velocity in a single node. Setting
the other quantities can easily be implemented.
Example:
\begin{lstlisting}[ numbers=none]
puts [ lbnode 0 0 0 set u 0.01 0. 0.]
\end{lstlisting}
\subsection*{The \lstinline|lb_boundary| command}
The \lstinline|lb_boundary| allows to set boundary conditions for the LB fluid. In general
periodic boundary conditions are applied in all directions, and only if LB boundaries
are constructed finite geometries are used. This part of the LB implementation is still experimental,
so please tell us about your experience with it. In general even the simple case of no-slip
boundary is still an important research topic in the lb community, and in combination with
point particle coupling not much experience exists. This means: Do research on that topic, play
around with parameters and find out what happens. 


The \lstinline|lb_boundary| command is supposed to resemble exactly the constraint command of 
\ES{}: Just replace the keyword \lstinline|constraint| with the word \lstinline|lb_boundary| 
and \ES{} will create walls with the same shape as the corresponding constraint. Example:
The commands
\begin{lstlisting}[ numbers=none]
lb_boundary wall 0.5 0 0 normal 1. 0. 0. 
lb_boundary wall -8.5 0 0 normal -1. 0. 0. 
\end{lstlisting}
create a channel with walls parallel to the $yz$ plane with width 8.

Currently only the so called \emph{link bounce back} method is implemented, where the effective
hydrodynamic boundary is located midway between two nodes. This is the simplest and yet a 
rather effective approach for boundary implementation. The \lstinline|lb_boundary| command
checks for every LB node if it is inside the constraint or outside and flags it as a boundary
node or not. This means if the lattice constant is set to 1, the above command yields exactly
the same as this:
\begin{lstlisting}[ numbers=none]
lb_boundary wall 0.1 0 0 normal 1. 0. 0. 
lb_boundary wall -8.1 0 0 normal -1. 0. 0. 
\end{lstlisting}
This has to be kept in mind, when you use the LB boundaries.

Currently only the shapes wall, sphere and cylinder are implemented, but to implement the others 
is straightforward.
