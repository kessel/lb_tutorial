

\section{The LB interface in \ES{}}
\ES{} has three main commands for the LB module: 
lbfluid, lbnode, and lb_boundary.
lbfluid is mainly used to set up parameters and does everything that
concerns the whole fluid. lbnode involves readout and manipulation of
single LB cells. lb_boundary allows to set boundaries, currently only
the bounce back boundary method is implemented to model
no-slip walls. All functions do quite exactly what one would 
assume they do, so we just cite the usage from the \ES{} manual here.

Important Notice: All commands of the LB interface use
MD units. This is convenient, as e.g. a particular 
viscosity can be set and the LB time step can be changed without
altering the viscosity. On the other hand this is a source
of a plethora of mistakes: The LBM is only reliable in a certain 
range of parameters (in LB units) and the unit conversion
may take some of them far out of this range. So not that you always
have to assure that you are not messing with that!

One brief example: a certain velocity may be 10 in MD units.
If the LB time step is 0.1 in MD units, and the lattice constant
is 1, then it corresponds to a velocity of 1 in LB units. 
This the maximum velocity of the discrete basis set and therefore
causes numerical instabilities, like negative populations.

\section{LB Bascis}
Set up an LB fluid with a 16x16x16 lattice. 
Set the temperature, viscosity and to 1.

All cells are initialized in equilibrium. Investigate the
equilibrium distribution in population space and in mode space.

Calculate the autocorrelation function of the stress tensor
Can you confirm the Landau-Lifshitz fluctuation formula

Now calculate the autocorrelation function 
