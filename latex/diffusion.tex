\chapter{The LBM in brief}
Here we want to repeat a few very basic facts about the LBM. 
You will find much better introductions in various books and
articles, e.g. \cite{succi, duenweg}. It will however help clarifying 
our choice of words and we will eventually say something about the 
implementation. It is very loosely written, with the goal that
the reader understands how the LBM works and what \ES{} does without
being precise with all details. 

The LBM essentially consists in solving a fully discretized
version of the linearized Boltzmann equation. The Boltzmann equation
describes the time evolution of the one particle distribution
function, which is the probability to find a molecule in a phase
space volume $\leftf(x,p\right)\d x \d p$. In the context of
LB is useful to apply a slightly different normalization of
the one particle distribution function: The function $f$ is normalized
so that the integral over the whole phase space is the total 
number of particles:
\begin{equation*}
  \int \leftf(x,p\right)\d x \d p = N,
\end{equation*}
so that the quantity $\leftf(x,p\right)\d x \d p$ corresponds
to the number of particles in this particular cell of the phase
space, the population. \\

The LBM discretizes the Boltzmann equation not only in real
space (the lattice!) and time, but also the velocity space is discretized 
in a surprisingly small number of velocities, in 3D typically
19, sometimes more, rarely less. 
Mostly we will refer to the three dimensional model with a discrete
set of 19 velocities, which is conventionally called D3Q19.
These velocities
are chosen so that they correspond the movement from one lattice
node another one. A two step scheme is used to transport
information through the system: In the streaming step
the particles ( in terms of populations ) are transported
to the cell where they corresponding velocity points to. 
In the collision step, the distribution functions
in each cell are relaxed towards the local thermodynamic
equilibrium. This will be described in more detail below.

The important hydrodynamic properties, the density, the fluid velocity,
the pressure tensor can be calculated quite straightforward from
the populations: They just correspond the to the moments
of the distribution function: 
\begin{align}
  \rho = \sum f_i \\
  \vec{u} = \sum f_i \vec{c_i} \\
  \vec{\Pi}^{\alpha \beta} = \sum f_i \vec{c_i}{\alpha}\vec{c_i}{\beta} \\
  \label{eq:fields}
\end{align}
We will sometimes call these quantities the hydrodynamic fields.
Note that the pressure tensor is symmetric.
It is easy to see that these equations are linear transformations
of the $f_i$ and that they carry the most important information. They
are 10 independent variables, but this is not enough to store the
full information of 19 populations. Therefore 9 additional quantities
are introduced. Together they form a different basis set of the
19-dimensional population space. This new basis set is called
the modes. Indeed the most practical basis is a little bit different:
It contains the trace of the pressure tensor and two orthogonal 
vectors instead of the three diagonal elements of the pressure 
tensor. The 9 extra modes are referred to as kinetic modes or
ghost modes.

The part that is now still missing is the collision part, where
the actual physics happens. For the LBM it is assumed that
the collision process linearly relaxes the populations to the local
equilibrium, thus that it is a linear (=matrix) operator 
acting on the populations in each LB cell. It should conserve 
the particle number and the momentum. At this point it is clear
why the mode space is helpful. A 19 dimensional matrix that
conserves the first 4 modes (with the eigenvalue 1) can just be written
down. Some struggle with lattice symmetries show that four independent
variables can be chosen to characterize the linear relaxation
process which are still compatible with the prerequisite that
the fluid should be isotropic: Two of them are closely related to 
the shear and bulk viscosity of the fluid, and two of them
do not have a direct physical equivalent. They are just called
relaxation rates of the kinetic model.


particle coupling \\

units \\


\chapter{Polymer Diffusion}
In these exercises we want to use the LBM-MD-Hybrid to reproduce a classic
result of polymer physics: The dependence of the diffusion coefficient
of a polymer on its chain length. If no hydrodynamic interactions
are not present, one expects a scaling law $D \propto N^{-1}$ and if 
they are present, a scaling law $D \propto N^{\nu}$ is expected. 
Here $\nu$ is the Flory exponent that plays a very prominent role
in polymer physics, which has a value of $\sim 3/5$ in good solvent
conditions. Discussions of these scaling laws can be found
in polymer physics textbooks like \cite{degennes, doi, rubinstein}.


We want to determine the diffusion coefficient from the mean square
distance that a particle travels in the time $t$. For large $t$ it should
be proportional to the time and the diffusion coefficient occurs as 
prefactor: 
\begin{equation}
  from Frenkelsmit
  \label{eq:msd}
\end{equation}
This equation can be found in virtually any simulation textbook, as 
\cite{frenkel}
We will therefore set up a polymer in an LB, simulate for an appropriate
amount of time, calculate the mean square displacement as a function of
time and obtain the diffusion coefficient from a linear fit. However
we make a couple of steps in between and divide the full problem in 
subproblems that allow to (hopefully) fully understand the process.

\section{Step 1: Diffusion of a single particle}
Our first step is to investigate the diffusion of a single particle
that is coupled to an LB fluid with the point coupling method.
Investigate the script single_particle_diffusion.tcl.

In this script an LB fluid and a single particle is created and LB is
used to thermostat the system. The random forces on the particle and
within the LB fluid will cause the particle to move, and its position
is recorded in the file pos.dat. Run the simulation script for 10000 steps
and use the helper script msd.pl to calculate the MSD. Use gnuplot
to investigate the curve:

plot pos.dat
plot msd_pos.dat

Where is the crossover between ballistic motion and brownian motion?
Use a linear fit to determine the diffusion coefficient:
f(x)=a*x+b
fit [10:] msd_pos.dat via a,b

Now change the box size from 16 to 8, 24, 32. What can you observe?
Can you explain your observation?
What happens if you replace the lb thermostat (and the LB fluid)
by a Langevin thermostat?

\section{Step 2: Understanding the method better}
Run the simulation again with different values for the friction
coefficient, e.g. 1. 5. 20. 50. Calculate the diffusion
coefficient for all cases and use gnuplot to make a plot of
$D$ as a function of $\gamma$. What do you observe?
The tiny helper script D.sh will help you with that. It contains
a (quite ugly) gnuplot one-liner that does the fitting and just
returns the slope. Now keep the friction coefficient fixed and
change the viscosity of the fluid. Make a plot.

\section{Step 3: The long time tail of the velocity autocorrelation function}

\section{Step 4: Setting up a polymer}

\section{Step 5: Setting up a self-avoiding polymer}

\section{Step 6: Diffusion of polymers}



