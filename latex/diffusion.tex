
\chapter{Polymer Diffusion}
In these exercises we want to use the LBM-MD-Hybrid to reproduce a classic
result of polymer physics: The dependence of the diffusion coefficient
of a polymer on its chain length. If no hydrodynamic interactions
are present, one expects a scaling law $D \propto N^{-1}$ and if 
they are present, a scaling law $D \propto N^{-\nu}$ is expected. 
Here $\nu$ is the Flory exponent that plays a very prominent role
in polymer physics. It has a value of $\sim 3/5$ in good solvent
conditions in 3D. Discussions of these scaling laws can be found
in polymer physics textbooks like \cite{degennes79a, doi96a, rubinstein03a}.


The reason for the different scaling law is the following:
When being transported, every monomer creates a flow field that follows
the direction of its motion. This flow field makes it easier 
for other monomers to follow its motion. This makes a polymer
long enough diffuse more like compact object including the fluid
inside it, although it does not have clear boundaries. It can be shown 
that its motion can be described by its hydrodynamic radius. It is defined 
as:
\begin{equation}
  \langle \frac{1}{R_h} \rangle = \langle \frac{1}{N^2}\sum_{i\neq j} \frac{1}{\left| r_i - r_j \right|} \rangle
\end{equation}
This hydrodynamic radius exhibits the scaling law  $R_h \propto N^{\nu}$
and the diffusion coefficient of long polymer is proportional to its inverse.
For shorter polymers there is a transition region. It can be described
by the Kirkwood-Zimm model:
\begin{equation}
  D=\frac{D_0}{N} + \frac{k_B T}{6 \pi \eta } \langle \frac{1}{R_h} \rangle
\end{equation}
Here $D_0$ is the monomer diffusion coefficient and $\eta$ the 
viscosity of the fluid. For a finite system size the second part of the
diffusion is subject of a $1/L$ finite size effect, because
hydrodynamic interactions are proportional to the inverse
distance and thus long ranged. It can be taken into account
by a correction:
\begin{equation}
  D=\frac{D_0}{N} + \frac{k_B T}{6 \pi \eta } \langle \frac{1}{R_h} \rangle \left( 1- \langle\frac{R_h}{L} \rangle \right)
  \label{kirkwood}
\end{equation}
It is quite difficult to prove this formula with good accuracy. It will 
need quite some computer time and a careful analysis. So please don't be
too disappointed if you don't manage to do so.


We want to determine the diffusion coefficient from the mean square
distance that a particle travels in the time $t$. For large $t$ it is
be proportional to the time and the diffusion coefficient occurs as 
prefactor: 
\begin{equation}
  \frac{\partial \langle r^2 \left(t\right)\rangle}{\partial t} = 2 d D. 
  \label{eq:msd}
\end{equation}
Here $d$ denotes the dimensionality of the system, in our case 3.
This equation can be found in virtually any simulation textbook, like
\cite{frenkel02b}.
We will therefore set up a polymer in an LB fluid, simulate for an appropriate
amount of time, calculate the mean square displacement as a function of
time and obtain the diffusion coefficient from a linear fit. However
we make a couple of steps in between and divide the full problem into 
subproblems that allow to (hopefully) fully understand the process.

\section{Step 1: Diffusion of a single particle}
Our first step is to investigate the diffusion of a single particle
that is coupled to an LB fluid by the point coupling method.
Investigate the script  \lstinline|single_particle_diffusion.tcl|.

In this script an LB fluid and a single particle are created and
thermalized.
The random forces on the particle and
within the LB fluid will cause the particle to move, and its position
is recorded in the file  \lstinline|pos.dat|. 
Run the simulation script for at least 10000 steps
and use the helper script  \lstinline|msd.pl| to calculate the MSD: 
{\vspace{0,2cm}\small
\begin{lstlisting}[numbers=none]
./msd.pl pos.dat
\end{lstlisting}\vspace{0,2cm}
}

Use  \lstinline|gnuplot|
to investigate the curve:
{\vspace{0,2cm}\small
\begin{lstlisting}[numbers=none]
plot "pos.dat"
plot "msd_pos.dat"
\end{lstlisting}\vspace{0,2cm}
}
\begin{figure}[h]
  \begin{center}
    \includegraphics{../figs/msd.pdf}
  \end{center}
  \caption{Mean square displacement of a single particle}
\end{figure}

What is different for short times than for long times?
Can you give an explanation for the parabolic shape at short
times?
Use a linear fit to determine the diffusion coefficient:
{\vspace{0,2cm}\small
\begin{lstlisting}[numbers=none]
f(x)=a*x+b
fit [1:] f(x) ``msd_pos.dat'' via a,b
\end{lstlisting}\vspace{0,2cm}
}
The square brackets in the fit command tell \lstinline{gnuplot}
only to use the range right of $x=1$ for the fit.

The file \lstinline|energy.dat| contains the kinetic energy of the
particle as a function of the elapsed simulation time. Investigate
it, by plotting it with gnuplot. Calculate the average value of 
the kinetic energy e.g. by fitting a constant function with gnuplot.
What value would you expect from a working thermostat?

Run the simulation again with different values for the friction
coefficient, e.g. 1. 2. 4. 10. Calculate the diffusion
coefficient for all cases and use gnuplot to make a plot of
$D$ as a function of $\gamma$. What do you observe?
The tiny helper script \lstinline|fit_lin.sh| 
(with argument \lstinline|msd_pos.dat|)
will help you with that. It contains
a (quite ugly) gnuplot one-liner that does the fitting and just
returns the slope. The fit is performed in the range 5 to 40 that
has proved to work good for runs of $\sim 100000$ steps. You have to 
modify the script to change that range.
Is there any difference between the
friction coefficient that you put in, and the diffusion coefficient
you obtain?

%\section{Step 3: The long time tail of the velocity autocorrelation function}
%Should we do anything here?

\section{Step 2: Diffusion of a polymer}
One of the typical applications of \ES{} is the simulation of polymer chains 
with a bead-spring-model. For this we need a repulsive interaction
between all beads, for which one usually takes a shifted and truncated
Lennard-Jones (so called Weeks-Chandler-Anderson) interaction, 
and additionally a bonded interaction between 
adjacent beads to hold the polymer together. You have already learned
that the command
{\vspace{0,2cm}\small
\begin{lstlisting}[numbers=none]
inter 0 0 lennard-jones 1. 1. 1.125 0.25 0. 
\end{lstlisting}\vspace{0,2cm}
}
creates a Lennard-Jones interaction with $\varepsilon=1.$ $\sigma=1.$
$r_\text{cut} = 1.125$ and $\varepsilon\text{shift}=0.25$ between particles
of type 0, the desired 
repulsive interaction. The command
{\vspace{0,2cm}\small
\begin{lstlisting}[numbers=none]
inter 0 FENE 7. 2. 
\end{lstlisting}\vspace{0,2cm}
}
creates a FENE (see \ES{} manual for the details) bond interaction. Still \ES{}
does not know between which beads this interaction should be applied.
This can be either be specified explicitly or done with the \lstinline|polymer|
command. This creates a given number of beads, links them with the given
bonded interaction and places them following a certain algorithm. We will
use the pruned self-avoiding walk: The monomers are set according 
to a pruned self-avoiding walk (in 3D) with a
fixed distance between adjacent bead positions. The syntax is:
{\vspace{0,2cm}\small
\begin{lstlisting}[numbers=none]
polymer $N_polymers $N_monomers 1.0 types 0 mode PSAW bond 0 
\end{lstlisting}\vspace{0,2cm}
}
Using a random walk to create a polymer causes trouble: The random walk may 
cross itself (or closely approach itself) and the LJ potential is very
steep. This would raise the potential energy enormously and would make
the monomers shoot through the simulation box. The pruned self-avoiding
walk should prevent that, but to be sure
we perform some MD steps with a capped LJ potential, this means 
forces above a certain threshold will be set to the threshold in order to prevent
the system from exploding. To see how this is done, look at the script 
\lstinline|polymer_diffusion.tcl|.
It contains a quite long warmup command so that also longer polymers
are possible. You can probably make it shorter. Also the runtime
can be reduced. You should find out about necessary number of steps
by yourself.

It is called in the following way:
{\vspace{0,2cm}\small
\begin{lstlisting}[numbers=none]
/path/to/Espresso polymer_diffusion $N_monomers  
\end{lstlisting}\vspace{0,2cm}
}
This allows to quickly change the number of monomers without editing 
the script. Change the variable  \lstinline|vmd_output| to yes to 
look at the diffusing polymer.
For the warmup a Langevin thermostat is used to keep the temperature constant.
You will have to add the LB command by yourself.

Run the script with a polymer of chain length 16 and look at the output files
which are identical to the output files of the 
\lstinline|single_particle_diffusion.tcl|
script. 
Use \lstinline|msd.pl| and \lstinline|fit_lin.sh| to calculate the diffusion
coefficient as a function of the chain length. If you are familiar with
shell scripting, write a script that automatically changes the chain length.
Additionally the script writes the file \lstinline|rh.dat| that contains the
instantaneous value of the hydrodynamic radius. The script \lstinline|average.sh|
allows to calculate its average value.

With the help of the single particle script now add the LB fluid and calculate
the MSD again. What do you observe? Do not forget to remove the Langevin 
thermostat after the warmup. This can be done with the command
{\vspace{0,2cm}\small
\begin{lstlisting}[numbers=none]
thermostat off
\end{lstlisting}\vspace{0,2cm}
} 
Can you confirm the behaviour of eq.~\ref{kirkwood} when a LB fluid is added?
Not very easy to answer is the question of the statistical accuracy of the
data. A good way is to run the same simulation several times so that it produced
statistically independent data and then calculate the standard deviation of
the mean, like in lab experiments.


To check the Kirkwood-Zimm formula (eq.~\ref{kirkwood}) we have used shell 
scripts of the following type:
{\vspace{0,2cm}\small
\begin{lstlisting}[numbers=none]
n="10 20 30 40 50 60 80 90 100"
for i in $n; do
  mkdir $i 
  cd $i
  ~/espresso-git/espresso/bin/Espresso ../polymer_diffusion.tcl $i
  cd ..
done
\end{lstlisting}\vspace{0,2cm}
} 
{\vspace{0,2cm}\small
\begin{lstlisting}[numbers=none]
n="10 20 30 40 50 60 80 90 100"
for i in $n; do
  cd $i
  ../../src/msd.pl pos.dat > /dev/null
  ../../src/acf_columnwise.pl v.dat > /dev/null
  a=`../../src/fit_lin.sh msd_pos.dat`
  b=`../../src/average.sh rh.dat`
  echo "$i $a $b"
  cd ..
done
\end{lstlisting}\vspace{0,2cm}
} 
Maybe this is a helpful starting point for you. 
