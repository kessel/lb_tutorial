
\chapter{Polymer Diffusion}
In these exercises we want to use the LBM-MD-Hybrid to reproduce a classic
result of polymer physics: The dependence of the diffusion coefficient
of a polymer on its chain length. If no hydrodynamic interactions
are not present, one expects a scaling law $D \propto N^{-1}$ and if 
they are present, a scaling law $D \propto N^{\nu}$ is expected. 
Here $\nu$ is the Flory exponent that plays a very prominent role
in polymer physics, which has a value of $\sim 3/5$ in good solvent
conditions. Discussions of these scaling laws can be found
in polymer physics textbooks like \cite{degennes, doi, rubinstein}.


We want to determine the diffusion coefficient from the mean square
distance that a particle travels in the time $t$. For large $t$ it should
be proportional to the time and the diffusion coefficient occurs as 
prefactor: 
\begin{equation}
  \frac{\partial \langle r^2 \left(t\right)\rangle}{\partial t} = 2 d D. 
  \label{eq:msd}
\end{equation}
Here $d$ denotes the dimensionality of the system, in our case 3.
This equation can be found in virtually any simulation textbook, as 
\cite{frenkel}
We will therefore set up a polymer in an LB, simulate for an appropriate
amount of time, calculate the mean square displacement as a function of
time and obtain the diffusion coefficient from a linear fit. However
we make a couple of steps in between and divide the full problem in 
subproblems that allow to (hopefully) fully understand the process.

\section{Step 1: Diffusion of a single particle}
Our first step is to investigate the diffusion of a single particle
that is coupled to an LB fluid with the point coupling method.
Investigate the script  \lstinline|single_particle_diffusion.tcl|.

In this script an LB fluid and a single particle is created and LB is
used to thermostat the system. The random forces on the particle and
within the LB fluid will cause the particle to move, and its position
is recorded in the file  \lstinline|pos.dat|. Run the simulation script for 10000 steps
and use the helper script  \lstinline|msd.pl| to calculate the MSD. Use  \lstinline|gnuplot|
to investigate the curve:

{\vspace{0,2cm}\small
\begin{lstlisting}[numbers=none]
plot pos.dat
plot msd_pos.dat
\end{lstlisting}\vspace{0,2cm}
}

Where is the crossover between ballistic motion and brownian motion?
Use a linear fit to determine the diffusion coefficient:
{\vspace{0,2cm}\small
\begin{lstlisting}[numbers=none]
f(x)=a*x+b
fit [10:] msd_pos.dat via a,b
\end{lstlisting}\vspace{0,2cm}
}

Now change the box size from 16 to 8, 24, 32. What can you observe?
Can you explain your observation?
What happens if you replace the lb thermostat (and the LB fluid)
by a Langevin thermostat?

The file \lstinline|energy.dat| contains the kinetic energy of the
particle as a function of the elapsed simulation time. Investigate
it, by plotting it with gnuplot. 

\section{Step 2: Understanding the method better}
Run the simulation again with different values for the friction
coefficient, e.g. 1. 5. 20. 50. Calculate the diffusion
coefficient for all cases and use gnuplot to make a plot of
$D$ as a function of $\gamma$. What do you observe?
The tiny helper script \lstinline|fit_lin.sh| will help you with that. It contains
a (quite ugly) gnuplot one-liner that does the fitting and just
returns the slope. Now keep the friction coefficient fixed and
change the viscosity of the fluid. Make a plot of the diffusion coefficient
as a function of the friction parameter.

\section{Step 3: The long time tail of the velocity autocorrelation function}

\section{Step 4: Setting up a polymer}
One of the typical application of \ES{} is the simulation of polymer chains 
with a bead-spring-model. For this we need a repulsive interaction
between all beads, for which one usually takes a shifted and truncated
Lennard-Jones (so called Weeks-Chandler-Anderson) interaction, 
and additionally a bonded interaction between 
adjacent beads to hold the polymer together. You have already learned
that the command
{\vspace{0,2cm}\small
\begin{lstlisting}[numbers=none]
inter 0 0 lennard-jones 1. 1. 1.225 0.25 0. 
\end{lstlisting}\vspace{0,2cm}
}
creates a Lennard-Jones interaction with $\varepsilon=1.$ $\sigma=1.$
$r_\text{cut} = 1.225$ and $\varepsilon\text{shift}=0.25$ between particles
of type 0, the desired 
repulsive interaction. The command
{\vspace{0,2cm}\small
\begin{lstlisting}[numbers=none]
inter 0 FENE 7. 2. 
\end{lstlisting}\vspace{0,2cm}
}
creates a FENE (see \ES{} for the details) bond interaction. Still \ES{}
does not know between which beads this interaction should be applied.
This can be either be specified explicitly or done with the \lstinline|polymer|
command. This creates a given number of beads, links them with the given
bonded interaction and places them following a certain algorithm. We will
use the random walk: The monomers are set according to a random walk (in 3D)
with the given distance between adjacent bead positions. The syntax is:
{\vspace{0,2cm}\small
\begin{lstlisting}[numbers=none]
polymer 
\end{lstlisting}\vspace{0,2cm}
}
Using a random walk to create a polymer causes trouble: The random walk may 
cross itself (or closely approach itself) and the LJ potential is very
steep. This would enormously raise the potential energy an would make
the monomers shoot through the simulation box. 
However perform some MD steps with a capped LJ potential, this means 
forces above a certain threshold will be set to the threshold in order to prevent
the system from exploding. To see how this is done, look at the script 
\lstinline|polymer_diffusion.tcl|.

Run the script with a polymer of chain length 16 and look at the output files
which are identical to the output files of the \lstinline|single_particle_diffusion.tcl|
script. Here a Langevin thermostat is used to keep the temperature constant.
Use \lstinline|msd.tcl| and \lstinline|fit_lin.sh| to calculate the diffusion
coefficient as a function of the chain length.


With the help of the single particle script now add the LB fluid and calculate
the MSD again. What do you observe?

You should see that the diffusion coefficient now follows a power law with
a different exponent than without the LB fluid.

